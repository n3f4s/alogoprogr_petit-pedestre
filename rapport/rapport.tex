\documentclass{rapport}

\title{Projet INFO3 S5 : ipolytech\_SI3\_algoprogr}
\author{%
	Mikaël FOURIER,
	Jean LABAT,
	Vincent BONNEVALLE
}
\date{}

\pdfinfo{%
	/Title    (algoprogr)
	/Author   (Mikaël FOURIER, Jean LABAT, Vincent BONNEVALLE)
	/Creator  (Vincent BONNEVALLE)
	/Producer (Vincent BONNEVALLE)
}

\makeindex


\begin{document}
	\maketitle
	\tableofcontents
	\chapter*{Introduction}
Le jeu Little Stars For Little Wars est un jeu de stratégie en temps réél dans lequel chaque
joueur a pour but de contrôler des planètes en gerant le deplacements de ses unités.
Le but de ce projet est de réaliser une intelligence artificielle capable de jouer à ce jeu, et qui soit la meilleure possible.
Pour servir cet objectif, nous avons donc du étudier les meilleures façon de representer le terrain ainsi que les unités dans une architecture adequat,
et nous avons également élaborer une stratégie d'action pour notre intelligence artificielle.

	\addcontentsline{toc}{chapter}{Introduction}
	\chapter{Architecture}
%TODO : presenter ça mieux, schéma ??
		\section{Stockage des données}
Comme le bot consiste en plusieurs fonctions appelées par le client, il a fallu
choisir un moyen de conserver les données. Nous avons choisi d'utiliser deux
variables globales : UUID et MATCHES. UUID contient l'uid du joueur.
Normallement, on n'y accède qu'en lecture après initialisation. MATCHES est un
dictionnaire associant un id de match à un objet Match (cf ci-dessous)
, ceci afin de permettre à plusieurs matches de se dérouler en parallèle.

La boucle principale de jeu a été simplifiée au maximum, la complexité étant
distribuée entre les autres modules. On commence par récupérer les changements
sur le plateau (message STATE) qu'on parse. Ensuite, on récupère l'objet Match
correspondant au match concerné, on le met à jour avec les nouvelles données
puis on lui demande de calculer la stratégie. Finalement, on encode cette
stratégie et on l'envoie au serveur. Chaque module est bien séparé : l'ajout des
messages GAMEOVER et ENDOFGAME n'ont nécessités des changements que dans
protocol.py, ainsi que de légers changements haut-niveau dans lolipooo.py. Si le
format des ordres à envoyer changeait, le module strategy.py ne serait
probablement pas modifié.

		\section{Parseur}
			\subsection{Structure générale du parseur}
Chaque type de message (INIT, STATE,...) est parsé par une fonction nommée
`parse\_<type>`. Toutes ces fonctions sauf parse\_register utilisent des regex
définies en constantes globales, avec des groupes nommés pour les données
interressantes. L'utilisation de la méthode `groupdict` des regex permet de
récupérer le résultat parsé sous forme de dictionnaire associant le groupe nommé
) la valeur. Le reste des fonctions consiste en divers bidouillages pour
transformer certains str en int par exemple.

Nous avons également mis en place un log simple utilisant le module `logging`.
Il consiste à afficher le message reçu puis le message parsé. Cela permet en cas
de problème que les parsing a été exécuté correctement.

Dans la boucle principale, le bot peut recevoir des messages STATE, ENDOFGAME ou
GAMEOVER. La fonction `parse\_message` est utilisée pour appeler la bonne
fonction de parsing. Le type de message est simplement reconnu à l'aide de la
méthode str.startwith. Une condition else permet de gérer le cas où un nouveau
type de message serait ajouté sans que le code du bot ne soit mis à jour. Le
tout est enrobé dans un try/catch pour attraper les erreurs sans faire crasher
le client.
			\subsection{Cas de parse\_init et de parse\_state}
Les messages de type INIT et STATE sont particuliers à traiter car ils
comportent des listes de taille variable. Les regex python ne permettant pas
d'avoir un nombre variable de groupes, nous avons dù diviser les listes et
parser individuellement chaque item. Par exemple, dans `parse\_state`, la string
des cellules est récupérée par la regex principale en un bloc. Celui-ci est
ensuite coupé au niveau des virgules, puis chaque fragment est parsé à l'aide de
la regex REGEX\_CELL\_STATE. La liste python résultante est ensuite réintroduite
dans le dictionnaire parsé à la place de la string.

Le champ CELLS du message INIT est plus délicat à parser, car il comporte une
virgule au sein même des items. La string est donc coupée au niveau de "I,",
puis "I" est rajouté à la fin de chaque cellule avant parsing. Ce n'est pas très
élégant, mais ça fonctionne.
			\subsection{Test}
Pour chaque type de message, un message d'example ainsi que le dictionnaire
parsé sont inclus dans le fichier source. Nous avons utilisé la fonctionnalité
doctest de python pour intégrer ces tests dans la docstring des fonctions. Les
docstrings sont assez limitées, le résultat attendu est récupéré par python sous
forme de texte. Comme les dictionnaires ne sont pas déterministes pour l'ordre
des clés, nous avons dû comparer le retour de la fonction avec le résultat
attendu et vérifier que cette comparaison est vraie, au lieu de simplement
mettre le dictionnaire en résultat.

		\section{Structure de données}
			\subsection{Introduction}
Nous avons choisi d'utiliser des classes pour l'encapsulation qu'elles
proposent. Voir le fichier lolipoop.py pour la simplicité de la boucle
principale.
			\subsection{Objet Match}
L'objet Match contient les informations relatives à un match. Celles-ci sont un
mélange de données statiques, comme les cellules et les liaisons entre elles, et
de données dynamiques, comme le nombre d'unités par cellule. De plus, chaque
objet Match contient une référence à la fonction calculant la stratégie. Cela
permet par exemple de choisir la stratégie à appliquer en fonction de
l'organisation des cellules.

Les données basiques, comprenant l'id du match et du joueur, la vitesse de jeu
et le nombre de joueurs, sont stockées sous leur forme d'origine, à savoir int
ou str. Le membre cells est plus complexe, il est stocké sous forme d'un tableau
associant un id de cellule à un objet Cell (cf ci-dessous). Ce tableau
permet de retrouver facilement l'objet Cell à l'aide de son identifiant.

Cet objet contient peu de logique. L'initialisateur se contente d'initialiser
les membres, les détails d'instanciation des cellules étant délégué à la classe
Cell. La mise à jour dynamique se fait simplement en parcourant les cellules et
en appelant la méthode update correspondante. Quand au calcul de la stratégie,
il est délégué à la fonction de stratégie choisie à l'initialisation.
			\subsection{Objet Cell}
Un objet Cell est instancié pour chaque cellule du plateau. Il centralise les
données relatives à cette cellule. Il contient des données simples, comme la
quantité maximale d'unités offensives, et des données plus complexes comme la
liste des mouvements à des mouvements à destination de la cellule ou un
dictionnaire associant l'id des cellules voisines à la distance les séparant de
la cellule courante.

La méthode update permet de mettre à jour les données dynamiques à partir des
données parsées dans la boucle principale.
			\subsection{Objet Movement}
L'objet Movement modélise un mouvement d'unités vers une cellule. Il stocke le
nombre d'unités en déplacement et leur propriétaire, la cellule de départ ainsi
que le temps restant avant arrivée des unités.

Nous avons choisi de stocker les déplacement dans la cellule d'arrivée car c'est
ce qui nous a paru être le plus utile.


	\chapter{Stratégie}
		Pour choisir la stratégie, il suffit de changer la valeur de la variable strategy\_name
		\section{Stratégies de base}
		Les stratégies \_strat\_base et \_strat\_base2 sont des stratégies basique : on attaque la cellule enemie adjacente la plus faible ou on aide la cellule adjacente la plus faible s'il n'y a pas de cellule enemie adjacente
		\section{\_less\_worse\_strat}
			Bien qu'assez basique, cette stratégie est un peu plus complexe que les stratégies de base.
			Le principe de cette stratégies est de partir des cellules ayant une cellule enemie adjacente, de leurs faire attaquer la cellule enemie adjacente la plus faible. Puis les cellules adjacentes aux cellules auquelles on vient de donner des ordres aident ces cellules.
			Puis on recommence jusqu'a que toutes les cellules aient des ordres
		\section{strat4}
			Le principe de cette stratégie est de lister les cellules par propriétaire et par
			danger (calculé par unit\_needed). On parcour la moitié la moins en danger de nos cellules puis on liste les actions possible (grâce à possible\_action) pour ces cellules.
			Puis on envoie un nombre d'unité correspondant au maximum entre 75\% des unités de la cellule source et le nombre d'unité attendu par la cellule cible.
		\section{strat5}
			\subsection{Fonctions utilisées dans la stratégie}
Cette strategie repose sur la fonction cell\_value qui permet de calculer "l'importance" d'une cellule
pour de determiner quelles sont les cellules que nous souhaitons capturer en priorité.
Cette fonction dépend de 2 parametres importants:
		-Le premier est éidement la production de la cellule.
		-Le second est la distance de la cellule à la cellule ennemie la plus proche en nombre de saut. Cette distance
permet de determiner si la cellule est proche des enemis ou pas, et donc si elle risque d'avoir besoin d'unités ou pas.
Cela permet d'envoyer nos unités en direction du front en priorité.
La deuxième fonction importante pour cette stratégie est la fonction unit\_needed, qui calcule un "danger" pour une cellule,
en fonction du nombre de cellules ennemies qu'il y a autour de la cellule, et des deplacements en direction de celle\_ci.
Ce taux de danger permet autant de savoir si une cellule ennemie est vulnerable que de savoir si une celue alliée elle permet donc
de savoir le nombre de renfort dont pourrait avoir besoin une cellule, mais aussi de determiner si on peut attaquer une cellule enemie.
			\subsection{Fonctionnement de la stratégie}
En se basant sur l'importance des cellules définie ci-dessus , on créé donc une liste des cellules trié par ordre d'importance,
On regarde ensuite chacune de nos cellules, et pour chacune d'entre elle, on va envoyer des unités aux cellules adjacentes qui en ont besoin
en en envoyant en priorité aux cellules les plus importantes.
On separe ici les cas selon le propriétaire de chaque cellule adjacente, car le nombre d'unités à envoyer peut varier selon le proprietaire.
Il a fallu trouver un moyen d'actualiser les nombre d'unités dont une cellule avait besoin avant que les ordres ne soient transmis et qu'un nouveau
state ne soit recupéré, c'est pourquoi nous avons ajouté l'atribut unit\_needed à la classe cellule. Nous definissons donc le nombre au debut de la boucle
en l'actualsant pour toutes les cellules puis nous l'actualisons au fur et à mesure que les ordres dont donnés.
		\section{strat6}
			Tout d'abord, cette stratégie consiste à établir une table de routage pour chaque cellule, cette table de routage consiste en une liste de ligne contenant la cellule cible, la cellule vers laquelle aller vers la cible, ainsi que la distance à parcourir par ce chemin. La table de routage est créé une seule fois.

			Ensuite, on séléctionne les cellules enemis ou neutre pouvant être attaquée. Puis on les répartis entre nos cellules les cibles.

			Enfin, les cellules attaquent leurs cibles en prenant la route donnée par leurs tables de routage.
			Le calcul de du nombre d'unité à envoyer ce fait ainsi:
			\begin{itemize}
					\item Si la cible est neutre
						\subitem Si les cellules adjacentes alliées on assez d'unité pour conquérir la cellule et la garder on attaque avec la moitié des unité
						\subitem Si sans nous, les cellules adjacentes alliées n'ont pas assez, on envoie les $\frac{3}{4}$ des unités de la cellule
						\subitem Sinon on envoie $\frac{1}{4}$ des unités
					\item Si la cellule est alliée
						\subitem Si on a pas assez pour l'aider, on envoie $\frac{3}{4}$ de nos unité
						\subitem Sinon on envoie le nombre d'unité dont elle a besoin
					\item Si la cellules est enemie
						\subitem Si on a un rapport nombre d'unité offensives / vitesse de production est le double de la cellule enemie : on attaque en force
						\subitem Si le rapport est supérieur à 1 et qu'on est aidé par les cellules alliée adjacentes, on fait de même.
						\subitem Sinon on envoie $\frac{1}{4}$
			\end{itemize}
	\chapter{Répartition des tâches}
		Nous nous sommes réparti les tâches en deux sous groupes. Une personne (Mikaël) s'est occupé de la communication
		avec le serveur. Deux personnes (Jean et Vincent) se sont occupé des stratégies. Les stratégies de bases (\_strat\_base, \_strat\_base2 ainsi que \_less\_worse\_strat) on été crées à deux. Pour les autres stratégies, nous avons décidé de nous séparer les tâches (entre Jean et Vincent) pour explorer différentes options.
	\chapter{Problèmes recontré}
		Nous n'avons pas réussi à transmettre les ordres au serveur qui ne les recevais pas. Nous n'avons donc pas pu tester convenablement nos I.A.
	\chapter*{Conclusion}
La programmation de cette IA a été une experience très enrichissante,elle nous a permit de découvrir le travail en équipe et la répartition des taches, 
il nous a également permit d'approfondir nos connaissances en python.

En effet, nous avons dû nous répartir les tâches de façon équitable : une personne ayant trop ou trop peu de travail aurait ralenti tout le groupe.
De plus, le projet nous à forcé à travailler collectivement, à discuter souvent de l'avancement du projet et ainsi travailler en équipe et non chacun 
de notre coté.

Enfin, ce projet nous à permi de découvrir le gestionnaire de version git.
	\addcontentsline{toc}{chapter}{Conclusion}
	\printindex
\end{document}
